 \documentclass[12pt]{article}
\usepackage{blindtext}
\usepackage{hyperref}
\usepackage{lineno}
\linenumbers

\renewcommand\thelinenumber{
\href{http://www.github.com/seneubert/latex-lineno-smart/blob/master/doc.tex\#L\the\inputlineno}{\arabic{linenumber}}
}

\begin{document}

%%%%##########################################################################

\title{Test Document -- smart line numbers}
\author{Sebastian Neubert}
\date{}
\maketitle
\thispagestyle{empty}
\pagestyle{empty}

%\begin{abstract}
%\end{abstract}
\blindtext

And here is a bit more of my own text to show better how this works. Let's try to run over a couple of lines here with some gibberish.
abddfdfgldkfjgh;sflghsdlfkgh;sdlkfgsddfgsdfgsdfgsdlfglsdkfjglakdjflgkjasdlfgjsldfjglksdjfgkljsdflkgjsldkfjglksdjflgjsdlfkgjlsdfjglksdjflgkjsdlfkjglsdkfjglksdjflgkjsdlfkgjsldkfjglsdjfglksjdflgkjsdlfkjglsdkfjglksdjfglksjdflkgj

Probably blindtext is not the best way to test this.

However manualr entry of lines makes it really come out nice.


\end{document}
