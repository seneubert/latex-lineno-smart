 \documentclass[12pt]{article}
\usepackage{blindtext}
\usepackage{hyperref}
\usepackage{lineno}
\linenumbers

\renewcommand\thelinenumber{
\href{http://www.github.com/seneubert/latex-lineno-smart/blob/master/doc.tex\#L\the\inputlineno}{\arabic{linenumber}}
}

\begin{document}

%%%%##########################################################################

\title{Test Document -- smart line numbers}
\author{Sebastian Neubert}
\date{}
\maketitle
\thispagestyle{empty}
\pagestyle{empty}

%\begin{abstract}
%\end{abstract}
\blindtext

And here is a bit more of my own text to show better how this works. Let's try to run over a couple of lines here. We will see that, a bit counterintuitive the links point to the end of the previous block of text. Probably because this is where the inputline counter is last set.
Probably blindtext is not the best way to test this.

However manual entry of lines makes it really come out nice.


\end{document}
